En este capítulo se indican todos los documentos y páginas que han sido usados como referencia durante el desarrollo del proyecto.

\begin{thebibliography}{11}

\bibitem{wiki:er}
Wikipedia Escritorio Remoto: \url{http://es.wikipedia.org/wiki/Escritorio_remoto}

\bibitem{wiki:vnc}
Wikipedia Virtual Network Computing: \url{http://en.wikipedia.org/wiki/VNC}

\bibitem{RFB:RFB}
Remote Frame Buffer Protocol: \url{http://www.realvnc.com/docs/rfbproto.pdf}

\bibitem{wiki:Android}
Wikipedia Android (operating system): \url{http://en.wikipedia.org/wiki/Android_(operating_system)}

\bibitem{SDK:SDK}
Software Development Kit:  \url{http://developer.android.com/sdk/index.html}

\bibitem{NDK:NDK}
Native Development Kit: \url{http://developer.android.com/tools/sdk/ndk/index.html}

\bibitem{wiki:jni}
Wikipedia Java Native Interface: \url{http://en.wikipedia.org/wiki/Java_Native_Interface}

\bibitem{tightpage:tight}
TightVNC, Tight encoding: \url{http://www.tightvnc.com/decoder.php}

\bibitem{wiresharkpage:wireshark}
WireShark: \url{http://www.wireshark.org/}

\bibitem{FSpage:FS}
Free Software: \url{http://www.fsf.org/about/what-is-free-software}

\bibitem{GPLpage:GPL}
General Public License: \url{http://www.gnu.org/licenses/gpl.html}

\bibitem{wiki:realvnc}
Wikipedia RealVNC: \url{http://es.wikipedia.org/wiki/RealVNC}

\bibitem{multivnc:multivnc}
MultiVNC: \url{https://play.google.com/store/apps/details?id=com.coboltforge.dontmind.multivnc&hl=es}

\bibitem{androidvnc:androidvnc}
Android-vnc-viewer: \url{http://code.google.com/p/android-vnc-viewer/}

\bibitem{sherlock:sherlock}
SherlockBar: \url{http://actionbarsherlock.com/}

\bibitem{holo:holo}
Iterfaz Holo: \url{http://developer.android.com/design/style/themes.html}

\end{thebibliography}
