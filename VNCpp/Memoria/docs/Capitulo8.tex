\section{Conclusiones}
Como conclusiones creemos que se han alcanzado los objetivos propuestos en el apartado 1.3. Todo el código ha sido licenciado bajo una licencia libre (GPL versión 3), por tanto el objetivo de crear una aplicación de software libre ha sido cumplido sin mayor problema. En cuanto a el objetivo más importante, la construcción de la aplicación de escritorio remoto en Android utilizando código nativo, también ha sido cumplido, ya que no sólo se ha construido una aplicación con muchas funcionalidades, si no que, es una aplicación eficiente, capaz de copetir con todas las demás aplicaciones de escritorio remoto para Android que hay ahora mismo. Cabe destacar una vez más que la gran diferencia y la razón por la que este proyecto es inovador, es en que ninguna de las demás aplicaciones utilizan código nativo, al menos no las de software libre. La imposibilidad de saber como están construidas las aplicaciones privativas nos impiden afirmar con total rotundidad que no exista ninguna aplicación de escritorio remoto para Android que utilice el NDK, no obstante, al no existir ninguna de software libre, el factor de inovación sigue estando ahí, ya que hemos trabajado sobre algo que no se podía encontrar en ningún lado con unos resultados considerablemente buenos.

\section{Trabajo Futuro}

Como posible trabajo futuro se podría pensar en portar la aplicación a las demás plataformas móviles como iOS o Firefox OS. 
