
El desarrollo de las redes de telecomunicaciones ha permitido que poco a poco el control remoto de un ordenador pase de meras terminales de texto a los escritorios remotos de hoy en día. La tecnología de escritorio remoto permite la centralización de aplicaciones que normalmente se ejecutarían en el entorno de usuario. Gracias a esta tecnología el entorno de usuario (cliente) se convierte en una simple terminal de entrada/salida. Todos los eventos de pulsación de teclas y movimientos del ratón se transmiten desde el cliente a un servidor donde la aplicación los procesa como si se tratasen de eventos locales. La imagen de la pantalla se devuelve al cliente y de esta manera se puede controlar desde el terminal cliente la máquina donde se encuentra la aplicación servidor.\\

Este proyecto se centra únicamente en la realización de una aplicación cliente para Android. De tal manera que la aplicación servidor queda fuera del dominio de este proyecto. Aunque más adelante se profundizará en cuales son los objetivos concretos de este proyecto, conviene aclarar que el objetivo era desarrollar un cliente VNC para Android, que utilizará el NDK y que fuera de software libre, ya que en la actualidad no existe ninguno que cumpla estos dos requisitos.\\

Existen diferentes protocolos de comunicación entre la aplicación cliente y la aplicación servidor. Este proyecto se ha desarrollado utilizando el protocolo de comunicación Remote Frame Buffer (RFB), utilizado por todos los clientes y servidores VNC. De esta manera la aplicación será compatible con cualquier servidor VNC instalado en cualquier terminal.\\

Antes de profundizar más en como se han afrontado los objetivos del proyecto conviene establecer ciertos conocimientos básicos relativos al dominio de estas aplicaciones. Comenzaremos explicando algunos conceptos que se han utilizado antes, tales como VNC, RFB, Android, etc.\\

\section{Conceptos}

\subsection{Escritorio remoto}

Un escritorio remoto es una tecnología que permite a un usuario controlar desde un terminal gráfico (cliente) otro terminal gráfico (servidor). X-Window fue el primer escritorio remoto desarrollado por el Massachusetts Institute of Technology (MIT) en la década de los ochenta \cite{wiki:er}.\\

\subsection{VNC}

VNC son las siglas en inglés de Virtual Network Computing \cite{wiki:vnc}. Es un sistema de escritorio remoto que utiliza el protocolo Remote Frame Buffer (RFB) para controlar remotamente otra máquina. Está basado en una arquitectura cliente-servidor, donde la máquina con la aplicación cliente es la que controla a la máquina con la aplicación servidor. VNC transmite los eventos de ratón y teclado desde el cliente al servidor y éste devuelve las actualizaciones de pantalla en la otra dirección a través de una red de comunicaciones.\\

VNC es independiente del sistema operativo, es decir, un cliente VNC ejecutado en una máquina Windows podría controlar un servidor ejecutado en una máquina Linux. En definitiva, se podría controlar cualquier sistema operativo desde cualquier sistema operativo siempre que ambos sean compatibles con VNC.\\

\subsection{RFB}

RFB son las siglas en inglés de Remote Frame Buffer. Es un protocolo de comunicación para el acceso remoto a interfaces gráficas de usuario. Como trabaja a nivel de framebuffer es aplicable a todos los sistemas de ventanas (X11, Windows, Macintosh). RFB es el protocolo usado en VNC \cite{RFB:RFB}.\\

\subsection{Android}

Android es un sistema operativo basado en Linux, diseñado principalmente para Smartphones y Tablets. Inicialmente fue desarrollado por Android Inc, pero finalmente Google lo compró en 2005. Fue presentado en 2007 junto con la fundación Open Handset Alliance y el primer móvil con sistema operativo Android salió al mercado en 2008.\\

La mayoría del código de Android está liberado bajo una licencia Apache, licencia libre y de código abierto. No obstante no está liberado completamente por lo tanto no se puede decir que Android sea un sistema operativo libre. Uno de sus puntos fuertes es la facilidad para desarrollar aplicaciones gracias a las herramientas que proporciona Google. Estas aplicaciones normalmente estan escritas en Java y se desarrollan utilizando el Android SDK, pero también se pueden construir aplicaciones escritas en C/C++ usando el Android NDK \cite{wiki:Android}.\\

\subsection{Android SDK}

SDK son las siglas en inglés de Software Development Kit. El SDK proporcina las APIs y herramientas de desarrollo necesarias para construir, testear y depurar aplicaciones para Android \cite{SDK:SDK}.\\

\subsection{Android NDK}

NDK son las siglas en inglés de Native Development Kit. El NDK es un conjunto de herramientas que permiten desarrollar parte de la aplicación para Android con lenguajes de código nativo como C o C++ \cite{NDK:NDK}.\\

\subsection{JNI}

JNI son las siglas en inglés de Java Native Interface. Es un framework de programación que permite a código Java llamar y ser llamado desde apliaciones nativas y librerias escritas en otro lenguajes como pueden ser C, C++ o ensamblador \cite{wiki:jni}. JNI proporciona todo lo necesario para que la comunicación entre el SDK y el NDK sea posible. Gracias a él es posible construir aplicaciones en Android utilizando código nativo.

\subsection{Tight encoding}

Tight es un tipo de codificación más compactada, es muy eficiente y está optimizado para conexiones con poco ancho de banda. Tight  incluye tanto la eficiencia de compresión sin pérdidas como opcionales formas de compresión con perdidas para JPEG \cite{tightpage:tight}.

\subsection{WireShark}

WireShark es un software analizador de protocolo de red (sniffer) licenciado bajo GPLv2 \cite{wiresharkpage:wireshark}. Es uno de los sniffer mas importantes y completos, en él contribuyen expertos en redes de todo el mundo.

\subsection{Free Software}

Free Software del inglés software libre. Es el software que permite al usuario compartir, estudiar y modificarlo \cite{FSpage:FS}. Todo sotware libre debe garantizar las libertades de usar el programa con cualquier propósito, estudiar y modificar el programa, distribuir copias del programa y por último la libertad de hacer públicas las modificaciones que se hayan realizado.

\subsection{GPL}

GPL son las siglas en inglés de General Public License \cite{GPLpage:GPL}. Es una licencia de código libre redactada por la Free Software Fundation. El código bajo esta licencia es código libre ya que esta licencia garantiza las cuatro libertades de éste.
\newpage
\section{Motivaciones}

La motivación principal para la realización de este proyecto era el desarrollo de algo nuevo, algo que no se hubiera hecho. Así, junto con nuestro interés por el desarrollo de aplicaciones Android usando código nativo, nos dimos cuenta de que no existía ninguna aplicación de escritorio remoto que usase esta tecnología. La realización de este proyecto suponía un desafío de cara a la interconxión entre el código en C++ y Java ya que para ello fue necesario aprender a hacer compilaciones cruzadas, a parte de aprender a utiliar JNI.

\section{Objetivos}

Los objectivos de este proyecto son principalmente dos, el desarrollo de una aplicación de escritorio remoto para Android usando el NDK y que dicha aplicación en su totalidad sea software libre.\\

El objetivo de realizar la aplicación no era sólo el hecho de construir una aplicación que funcionase correctamente. La aplicación tenía que ser usable y eficiente. Este objetivo lo conseguimos sobradamente ya que gracias al uso de código nativo se ha conseguido que la aplicación no sólo sea eficiente si no que es lo suficiente rápida como para poder competir con la aplicación oficial de los inventores del protocolo. En cuanto a la usabilidad, la aplicación proporciona muchas opciones como, por ejemplo, establecer conexiones cifradas y enviar combinaciones de teclas.\\

En cuanto al software libre, todo el código ha sido licenciado bajo una licencia GPL versión 3, por lo tanto el software de la aplicación garantiza las cuatro libertades que todo software debe tener para ser considerado como tal.\\

Consideremos importante y por eso esta en esta sección de objetivos, el que sea software libre, ya que creemos que el software libre supone un avance muy grande con respecto al mundo del software privativo, creando una comunidad de usuarios y programadores que interactúan entre sí construyendo mejor software. 
