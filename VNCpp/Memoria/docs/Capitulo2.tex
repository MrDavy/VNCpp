Como referencias actuales para realizar este proyecto, y coger ideas de aplicaciones enfocadas en el mismo campo, hemos usado:
\begin{itemize}
\item \textbf{VNC Viewer(RealVNC):} perteneciente a la compañía RealVNC. Inicialmente VNC surgió en el Olivetti \& Oracle Research Lab (ORL3), siendo desarrollado por Tristan Richardson (inventor), Andy Harter (director del proyecto), Quentin Stanfford-Fraser y James Weatherall. Unos años después, varios de estos desarrolladores crearían RealVNC con la idea de seguir con este proyecto software. Al no ser una aplicación de código abierto, es difícil saber el tipo de implementación que lleva a cabo \cite{wiki:realvnc}.
\item \textbf{Android-vnc-viewer:} es un programa libre y de código abierto de control remoto para dispositivos Android. Es un proyecto desarrollado a partir de una rama inicial de otra aplicación llamada tightVNC. Esta bajo licencia GPLv2 y tiene un repositorio activo en google \url{http://code.google.com/p/android-vnc-viewer/}.\cite{androidvnc:androidvnc}.
\item \textbf{MultiVNC:} es un proyecto de software libre bajo licencia GPLv2,  basado en android-vnc-viewer y desarrollado por Christian Beier. Incluye soporte para varias codificaciones, así como el uso de Zeroconf (Zero Configuration Networking) para ayudar a crear de forma automática una red IP sin necesidad de configuraciones o servidores especiales \cite{multivnc:multivnc}.
\end{itemize}
