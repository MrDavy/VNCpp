Como referencias actuales para realizar este proyecto, y coger ideas de aplicaciones enfocadas en el mismo campo, hemos usado:
\begin{itemize}
\item \textbf{VNC Viewer(RealVNC)\cite{wiki:realvnc}:} perteneciente a la compañía RealVNC. Inicialmente VNC surgió en el Olivetti \& Oracle Research Lab (ORL3), siendo desarrollado por Tristan Richardson (inventor), Andy Harter (director del proyecto), Quentin Stanfford-Fraser y James Weatherall. Unos años después, varios de estos desarrolladores crearían RealVNC con la idea de seguir con este proyecto software. Al no ser una aplicación de código abierto, es difícil saber el tipo de implementación que lleva a cabo.\\

Posee una interfaz gráfica muy usable y un rendimiento óptimo. Para el control del ordenador utiliza la pantalla a modo de touchpad, esto es, el movimientos en la pantalla son los movimientos que hará el ratón. Esto proporciona un manejo lento pero extremadamente preciso.\\

Por otro lado RealVNC no almacena las constraseñas sino que pregunta al usuario por ella cuando conecta a un servidor que la requiera.\\

Ha sido una de las aplicaciones de referencia, tanto por su interfaz como por su óptimo rendimiento. Como ya se ha dicho con anterioridad es de código cerrado, así que fue imposible trabajar sobre ella.

\item \textbf{Android-vnc-viewer(AndroidVNC)\cite{androidvnc:androidvnc}:} es un programa libre y de código abierto de control remoto para dispositivos Android. Es un proyecto desarrollado a partir de una rama inicial de otra aplicación llamada tightVNC. Esta bajo licencia GPLv2 y tiene un repositorio activo en google \url{http://code.google.com/p/android-vnc-viewer/}.\\

Posee una interfaz gráfica pobre y una gestión de conexiones poco clara. Para el control remoto del ordenador ofrece una amplio abanico de posibilidades cubriendo cualquier posibilidad al respecto.\\

En el apartado de la seguridad AndroidVNC almcena las contraseñas en la Base de Datos sin cifrar siendo vulnerable a ataques,\\

Al ser una aplicación de sofware libre hemos podio ver y aprender mucho de su código y ha sido una de las aplicaciones de referencia ya que ofrece un rendimiento muy bueno. Nos fue imposible trabajar sobre él ya que está construida totalmente sobre SDK y nuestro objetivo es utilizar la potencia del NDK.

\item \textbf{MultiVNC\cite{multivnc:multivnc}:} es un proyecto de software libre bajo licencia GPLv2,  basado en android-vnc-viewer y desarrollado por Christian Beier. Incluye soporte para varias codificaciones, así como el uso de Zeroconf (Zero Configuration Networking) para ayudar a crear de forma automática una red IP sin necesidad de configuraciones o servidores especiales.

Al igual que AndroidVNC posee una interfaz pobre y, aunque mejor, una gestión de conexiones no demasiado claro. Para la representación utiliza aceleración por hardware mediante GPU. Respecto al control utiliza un sistema similar a RealVNC.\\

En el aspecto de las contraseña presenta un funcionamiento semejante a AndroidVNC.\\

Esta aplicación ha tenido una relevancia menor en el proyecto ya que no nos convencía su rendimiento en móviles de baja gama, se ha usado para observar como resolvían ciertos aspectos y como comparativa de rendimiento.
 
\end{itemize}
